\documentclass[11pt,a4paper,titlepage]{article}
\usepackage[latin1]{inputenc}
\usepackage{booktabs}
\usepackage[normalem]{ulem}
\usepackage{amsmath}
\usepackage{amsfonts}
\usepackage{amssymb}
\usepackage{ upgreek }
\usepackage{lipsum}
\usepackage{mathtools}
\usepackage{todonotes}
\usepackage{graphicx}
\usepackage{float}
\usepackage{wrapfig}
\usepackage{caption}
\usepackage{subcaption}
%\usepackage{fullpage}
\usepackage{polyglossia}
\setdefaultlanguage{latvian}
\DeclarePairedDelimiter\abs{\lvert}{\rvert}%
\DeclarePairedDelimiter\norm{\lVert}{\rVert}%
\makeatletter
\let\oldabs\abs
\def\abs{\@ifstar{\oldabs}{\oldabs*}}
\let\oldnorm\norm
\def\norm{\@ifstar{\oldnorm}{\oldnorm*}}
\makeatother

\begin{document}
\begin{center}
  \begin{Huge}
  \textbf{Putraimdesas uzbrūk \\} 
\end{Huge}   {\Large Noteikumu grāmata}
\end{center}  
Atkal Vikonta Brinzona pilij uzbrūk putraimdesas. Pēc ziņnešu teiktā šoreiz tās ir sapulcējušās lielākā barā nekā 
parasti turkāt dabūušas arī asinsdesu algotņus. Varētu likties ka no iepriekšējiem sirojumiem pret Vikonta pils 
iespaidīgajiem nocietinājumiem putraimdesas būtu guvušas labu mācību, bet nelaimīgā kārtā šoreiz tās uzbrūk tieši 
tad, kad labākie un uzticamākie siera desu karotāji ir devušies kampaņā pret pīrāgu ostām dienvidos. Tādas nejaušības 
noteikti nav nejaušības - Vikonta karavīru starpā ir spiegi!

\section*{Spēles komponentes}
\begin{itemize}
\item Spēles laukums
\item 6 spēlētāju figūriņas
\item 6 lomu kārtis (4 siera desas un 2 putraimdesas)
\item 3 virsnieku žetoni (leitnants, sardzes kapteinis, seržants)
\item 9 nocietinājumu kārtis (+8, +4, 0)
\item 72 putraimdesu uzbrukuma kārtis.
\item Šķēršļu joslas kārtis
\end{itemize}

\section*{Spēles uzstādīšana}
Novieto spēlētāju figūriņas laukuma centrā pie karoga. Katrā nocietinājumu frontē novieto 3 nocietinājumu kārtis tā, lai
augšā atrastos +4 pa vidu +8 un apakšā 0. Izdala virsnieku žetonus pēc nejaušības principa tā, lai nevienam
spēlētājam nebūtu divi žetoni. Tiek sagatavotas sešas vienādas kaudzītes ar putraimdesu uzbrukuma kārtīm tā,
lai katrā būtu 12 kārtis. Katram spēlētājam iedod vienu kaudzīti. Katram spēlētājam izdala vienu lomas kārti.
\todo{iespējams spiegiem pirms spēles vajag iepazīties}

\section*{Spēles pārskats}
Spēle sastāv no 3 spēles raundiem un noslēguma raunda. Spēles raunds sastāv no 4 sekojošām fāzēm:
\begin{enumerate}
\item Darbu sadale
\item Darbu izpilde
\item Saziņa ar putraimdesu karavīriem
\item Virsnieku žetonu maiņa
\end{enumerate}
Noslēguma raundā arī ir 4 fāzes, bet pēdējās divas ir atšķirīgas:
\begin{enumerate}
\item Darbu sadale
\item Darbu izpilde
\item Nocietinājumu aizstāvju izvēlēšanās
\item Lielā kauja
\end{enumerate}

\section*{Darbu sadale}
Darbu sadales sākumā visas spēlētāju figūriņas tiek novietotas laukuma centrā pie karoga. Ja spēlētāja figūriņa stāv pie 
karoga tad tas nozīmē ka šo spēlētāju var norīkot. Tagad virsniekiem ir iespēja norīkot spēlētājus kuri stāv pie karoga. 
Ja virsnieka figūriņa stāv pie karoga tad atiecīgais spēlētājs var to novietot pilī un citu spēlētāju figūriņas novietot
pa posteņiem vai arī atstāt pie karoga, bet ja virsnieks pats jau ir norīkots tad viņš ir spiests neko nedarīt.
Vispirms iespēja komandēt citus spēlētājus tiek dota leitnantam, tad sardzes kapteinim un beigās seržantam. 

Ja leitnants izvēlas norīkot spēlētājus pie nocietinājumiem tad viņš noliek savu figūriņu pilī
un citu spēlētāju figūriņas noliek pie nocietinājumiem vai atstāj pie karoga. Pie katra nocietinājuma nedrīkst novietot
vairāk kā vienu spēlētāju. Ja leitnants izvēlas nenorīkot 
citus spēlētājus tad viņš arī pats savu figūriņu atstāj pie karoga un tad citiem virsniekiem ir iespēja norīkot viņa
figūriņu naktssardzei vai trenniņiem. 

Pēc tam rīkojas sardzes kapteinis. Viņš norīko cilvēkus sargtornī un sakārto secībā kurā tie pildīs savas darbības.
Ja sardzes kaptieņa figūriņa vairs neatrodas pie karoga tad viņš ir spiests neko nedarīt. Sardzes kapteinis var norīkot 
jebkuru figūru kura joprojām atrodas pie karoga, ieliekot savu figūru pilī.

Līdzīgā veidā rīkojas seržants norīkojot cilvēkus pie šķēršļu joslas. Arī seržants var izvēlēties neko nedarīt, kaut 
gan tā rīkotos tikai putraimdesu spiegs.

{Pils ir vieta kurā siera desu virsnieki nodarbojas ar birokrātiju. Nepieredzējušajām un plānādainajām siera desām
birokrātija ļoti nepatīk, jo viņi to nesaprot un dažādo atskaišu aizpildīšana tām aizņem visu dienu. Vikonts
tomēr uzstāj ka birokrātija ir jāpilda, jo tā ir sena dzimtas tradīcija, kaut arī pašam Vikontam birokrātija diez ko
nepatīk. Patiesībā birokrātiju izdomāja tikai pirms simts gadiem bariņš virsnieku kuri negribēja iet uz šķēršļu joslu.}

\section*{Darbu izpilde}
Darbu izpilde notiek tāda pat secībā kā norīkošana - vispirms rīkojas spēlētāji pie nocietinājumiem, pēc tam sargtornī
norīkotie un beigās tie kuri ir šķēršļu joslā.
  \paragraph*{Nocietinājumi: } Visi spēlētāji kuri norīkoti pie nocietinājumiem rīkojas vienlaicīgi. Katrs paņem sev
  atbilstošo nocietinājumu kaudzīti, tad slepus paskatās, tad slepus apmaina vietām divas augšējās kārtis, vai
  divas apakšējās kārtis vai arī nedara neko. Pēc tam novieto kaudzīti atpakaļ pie atbilstošā nocietinājuma.
  Lielajā kaujā tiks ņemta vērā tikai augšējā kārts.
  \paragraph*{Sargtornis: } Spēlētāji rīkojas secībā sākot ar pirmo. Spēlētājs sargtornī var darīt vienu no divām lietām.
  1.) Slepus aplūkot kārtis kādā no nocietinājumu kaudzītēm, bet nemainīt to secību.
  2.) Rūpīgi samaisīt putraimdesu uzbrukuma kārtis kuras atrodas mežā un paņemt sev trīs, un tās slepus aplūkot.
  \paragraph*{Šķēršļu josla: } Spēlētāji rīkojas secībā sākot ar pirmo. Katrs spēlētājs paņem šķēršļu joslas kārtis, 
  no tām viņš izvēlas vienu, atklāj pārējiem \todo{vēl jāizdomā vai atklāti, vai aizklāti}
  un patur sev priekšā. Pēc tam šķēršļu joslas kārtis tiek padotas 
  nākamajam spēlētājam. Kārts bonuss skaitīsies tikai tad ja dotais spēlētājs
  tiks izvēlēts lielajai kaujai un tikai tad ja viņš nav spiegs.
\\
Pēc katra spēlētāja darbības spēlētājiem var rasties vajadzība apspriesties un pastāstīt ko viņi ir redzējuši, bet 
nedrīkstēja rādīt citiem. Spēlētājiem ir atļauts apspriesties jebkurā brīdī ar vienu izņēmumu - nocietinājumu 
labošana notiek vienlaicīgi un tajā laikā neviens nerunā, tomēr, tieši pirms un tieši pēc tam spēlētāji drīkst 
apspriesties.

\section*{Saziņa ar putraimdesu karavīriem}
Katrs spēlētājs ieliek mežā vienu putraimdesu uzbrukuma kārti ar attēlu uz leju. Siera desu spēlētājiem būs  
jāieliek kārts kura neko nedara, bet ir svarīgi, lai visi spēlētāji paskatās uz savām kārtīm un brīdi padomā par to
ko viņiem likt, jo citādi uzreiz būtu skaidrs kuri ir spiegi. Putraimdesu spiegiem tikmēr ir iespēja izvēlēties ko likt, 
ieteicamā stratēģija ir censties uzbrukt no vienas puses vai uzbrukt tur, kur ir vāji nocietinājumi.

\section*{Virsnieku žetonu maiņa}
Vispirms leitnants iedod savu žetonu spēlētājam kuram nav žetona. Pēc tam sardzes kapteinis iedod savu žetonu spēlētājam
kuram nav žetona, turklāt viņš drīkst dot savu žetonu tam spēlētājam kurš pirms tam bija leitnants. Līdzīgi rīkojas
arī seržants. Šajā fāzē tiek pabīdīts uz priekšu putraimdesu karaspēks, lai spēlētāji atcerētos cik raundus ir 
izspēlējuši.

\section*{Nocietinājumu aizstāvju izvēlēšanās}
Kad ir izpēlēti pirmie trīs spēles raundi un noslēguma raunda pirmās divas fāzes, un putraimdesu karaspēks ir ieradies
sākas nocietinājumu aizstāvju ievēlēšana. Par nocietinājumu aizstāvjiem tiek ievēlēti divi spēlētāji. Šiem spēlētājiem
priekšā noliktās šķēršļu joslas kārtis tiks ņemtas vērā nosakot kaujas rezultātu. Spēlētāju pāris skaitās ievēlēts
tad, ja par tiem baslo vairāk kā puse spēlētāju. 
\section*{Lielā kauja}
Pirms lielās kaujas \todo{liesās kaujas} visas putraimdesu uzbrukuma kārtis kuras spēlētāji bija ieguvuši ar sargtorņa
palīdzību tiek atgriestas atpakaļ mežā. Pēc tam visas kārtis no meža tiek atklātas un katrā pils frontē tiek saskaitīts
cik putraimdesas tai uzbrūk. Tiek atklāta augšējā kārts no katras nocietinājumu frontes. Tiek atklātas visu spēlētāju
lomu kārtis. Ja kādā no kāršu atklāšanas brīžiem ir iespējams ieturēt spriedzes pilnu pauzi intrigas uzturēšanai tad
tas noteikti ir jādara.

Siera desu spēks katrā frontē ir vienāds ar attiecīgās frontes 
nocietinājumu vērtību, plus, abu aizstāvju šķēršļu joslas kārtīm, bet ja atklājas ka kāds no spēlētājiem ir spiegs
tad viņa kāršu vērtība netiek ņemta vērā un tā vietā aizsargi dabū -2 pie savas summas. Ja frontei uzbrūk vairāk 
putraimdesas nekā ir aizsargu spēks tad katra nepieveiktā putraimdesa tiek pie Vikonta pils izlaupīšanas.

Pa visām frontēm tiek saskaitītas putraimdesas kuras tika cauri nocietinājumiem un spiegi dabū katrs tik daudz punktus.
Ja siera desu aizsardzībai neizlauzās cauri neviena putraimdesa tad katra siera desa dabū 2 punktus.

Uzvar tie spēlētāji kuri pēc trīs spēlēm ir dabūjuši visvairāk punktus.
\end{document}

Kāpēc spiegi netiek likti cietumā tad kad ir skaidrs kuri viņi ir.

Kāpēc seržants var pavēlēt leitnantam
